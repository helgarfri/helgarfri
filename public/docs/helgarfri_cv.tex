\documentclass[11pt]{article}

% ------------------------------------------------------------
% PAKKAR
% ------------------------------------------------------------
\usepackage[a4paper,margin=1in]{geometry}  % Adjust margins as needed
\usepackage[utf8]{inputenc}                % UTF-8 input encoding
\usepackage[T1]{fontenc}                   % Good font encoding
\usepackage{lmodern}                       % Latin Modern fonts
\usepackage{inconsolata}  % Alternative monospaced font
\renewcommand{\familydefault}{\ttdefault}  % Set monospaced as default
\usepackage{ragged2e}                      % For ragged-right
\usepackage{xcolor}                        % For text colors if needed
\usepackage{hyperref}                      % For clickable links
\usepackage{fontawesome5}

\pagenumbering{gobble}

\RaggedRight

\newcommand{\cvsection}[1]{\vspace{2em}\textbf{\large #1}\par\vspace{1em}}

% ------------------------------------------------------------
% SRKÁ
% ------------------------------------------------------------
\begin{document}

% ------------------------------------------------------------
% HEDDERINN
% ------------------------------------------------------------
{\huge HELGI FREYR DAVÍÐSSON}\\
\medskip

\faGlobe\ \href{https://helgarfri.is}{helgarfri.is} \textbar\
\faEnvelope\ \href{mailto:helgifreyr02@gmail.com}{helgifreyr02@gmail.com} \textbar\
\faPhone\ +34 610 31 68 71 \textbar\
\faGithub\ \href{https://github.com/helgarfri}{helgarfri} \textbar\
\faInstagram\ \href{https://instagram.com/helgarfri}{helgarfri} \textbar\
\faLinkedin\ \href{https://www.linkedin.com/in/helgi-freyr-davíðsson-9841ba219}{helgi-freyr-davíðsson}


% ------------------------------------------------------------
% KYNNING
% ------------------------------------------------------------
\cvsection{stutt kynning}
ég er forritari og verðandi tölvunarfræðingur, með áherslu á vefþróun.
ég hef reynslu af því að byggja upp heila vefþjónustu frá grunni með Express.js og PostgreSQL. mér finnst gaman að leysa vandamál og langar að vinna við krefjandi verkefni tengd gagnavinnslu, hugbúnaðarþróun eða gagnagreiningu.

% ------------------------------------------------------------
% MENNTUN
% ------------------------------------------------------------
\cvsection{menntun}
b.s. í tölvunarfræði -- háskóli íslands \hfill (áætlað ústskriftarár: 2026)

\begin{itemize}
  \item ects: 84 / 180
\end{itemize}

stúdent -- framhaldskólinn í mosfellsbæ \hfill (2018 - 2022)

% ------------------------------------------------------------
% REYNSLA
% ------------------------------------------------------------
\cvsection{starfsreynsla}
\textbf{áfyllari} @ ölegerðin egill skallagrímsson \hfill (2023 - 2025)\\
- sá um áfyllingu, pantaði vörur í gegnum birgðakerfi og tryggði stöðugt framboð drykkjarvara.\\

\textbf{móttökustjóri} @ golfklúbbur mosfellsbæjar \hfill (2019 - 2023)\\
- sá um móttöku klúbbfélaga, sölu á vörum, og samskipti við gesti. \\
- notaði tímapantanakerfi fyrir rástíma, hélt utan um félagaskráningar og tryggði hnökralausa þjónustu.

% ------------------------------------------------------------
% HÆFNI
% ------------------------------------------------------------
\cvsection{hæfni}
\begin{itemize}
  \item \textbf{forritunarmál:} javascript, python, java, c, r, sql, html, css, kotlin, typescript, swift, latex
  \item \textbf{frameworks og libraries:} react.js, next.js, express.js, django, pandas, numpy
  \item \textbf{græjur og tól:} git/github, docker, linux/unix, macos, vs code, supabase, netlify, postgresql/mysql, aws
  \item \textbf{concepts og aðferðir:} object-oriented programming (oop), functional programming (fp), data structures \& algorithms, rest api development, database management, version control \& ci/cd, agile methodology
  \item \textbf{tungumál:} íslenska (móðurmál), enska (c2), spænska (b1)
\end{itemize}

\newpage

% ------------------------------------------------------------
% VERKEFNI
% ------------------------------------------------------------
\cvsection{verkefni}
\textbf{map in color} \hfill \href{https://mapincolor.com}{mapincolor.com}\\
verkefnið hefur verið í stöðugri þróun í nokkur ár. upphaflega var markmiðið að búa til forrit sem gerir notendum kleift að búa til kort út frá gögnum. þetta verkefni varð vendipunktur í minni þróun sem forritari, þar sem ég dýpkaði skilning minn á react og framendaforritun. þegar forritið þróaðist ákvað ég að taka það skrefinu lengra og bæta við bakenda með netþjóni, gagnagrunni og ýmsum viðbótum.

í dag geta notendur hlaðið upp csv-skrám, sérsniðið sín eigin kort og vistað þau á sínum svæðum. þeir geta deilt kortum með öðrum, haft gagnvirk samskipti við notendur, og greint gögn á sjónrænan hátt. markmið verkefnisins er að búa til vettvang þar sem hægt er að skoða og greina gögn í gegnum kort, hvort sem það er fyrir vísindalega úrvinnslu eða einfaldlega til að sjá heiminn í nýju ljósi út frá gögnum.\\
\textbf{Tækni notuð:} react.js, node.js, express.js, supabase, postgreSQL, git/gitHub, netlify.

\end{document}
